\documentclass[a4paper]{article}

\usepackage{graphicx}
\usepackage[colorlinks=true]{hyperref}
\usepackage{paralist}

\begin{document}



\title{PACT: \\ {\large Programming ${\land}$ Algorithms $\Rightarrow$ Computational Thinking}}

\author{Aidan Mooney
  \and James Power
  \and Rosemary Monahan
  \and Tom Naughton
  \and Susan Bergin
  \and Phil Maguire
  \and Joe Duffin
}

\begin{figure}\centering
\includegraphics[height=.2\textheight]{nuim_large}
\end{figure}

\maketitle

\section{Introduction}
The PACT programme is a partnership between researchers in the Department of Computer Science at NUI Maynooth and teachers at selected post primary schools around the country.   Starting in September 2013 a number of Irish secondary schools took part in a pilot study, delivering material prepared by the PACT team to transition year students.

Initially the focus of this partnership is on teaching programming to transition year students, but ultimately the goal is to develop a framework for delivering a short course on \textit{computational thinking} as part of the new Junior Certificate cycle.


\section{Motivation}

NUI Maynooth has been delivering programmes in Computer Science at undergraduate and postgraduate level for 25 years and, since 2012, has been delivering a unique \textit{BSc in Computational Thinking}.  This degree is a blend of Computer Science, Mathematics and Philosophy, and aims to provide a deeper education in computing than the more traditional degrees, which often have a strong vocational orientation.  The PACT initiative aims to build on the visibility of the \textit{Computational Thinking} brand to bring the ideas at the heart of this degree to second-level students.

\paragraph{What is Computational Thinking?}

The phrase \textit{Computational Thinking} was originally coined in the context of mathematics education by by Seymour Papert \cite{papert96}, but came to prominence in Computer Science following an influential article by Jeannette M. Wing \cite{wing-cacm06}.  Wing's article described computational thinking as
\begin{quotation}
``Computational thinking
builds on the power and
limits of computing
processes, whether they are executed by a human or by a
machine.  Computational thinking confronts the riddle of machine intelligence: What can humans do better than computers? and What can computers do better than humans? Most fundamentally it addresses the question: What is
computable?''
\end{quotation}


This perspective has been developed by the 
\textit{Center for Computational Thinking} at Carnegie Mellon University\footnote{\url{http://www.cs.cmu.edu/~CompThink/}}
(sponsored by Microsoft Research), and echoed in other programmes, including those by 
Google\footnote{\url{http://www.google.com/edu/computational-thinking/}} and the
\textit{International Society for Technology in Education}\footnote{\url{http://www.iste.org/learn/computational-thinking}}.

Crucial elements of this perspective are that:
\begin{itemize}
\item Computational thinking is a way that humans, not computers, think.
\item Computational thinking is a fundamental skill for
everyone, not just for computer scientists.
\end{itemize}


\paragraph{Computational Thinking in Schools.}

There have been numerous efforts over the years to introduce computational concepts into Irish secondary schools but, at an official level, these never proceeded much beyond basic elements of information technology.  The reform of the Junior Certificate cycle offers an opportunity to get this right: a course designed by Computer Scientists to display the depth and beauty of the field in a way that can challenge and engage second-level students.  An emphasis on 
\textit{Computational Thinking} allows us to explore the key concepts the underlie Computer Science, without necessarily having to achieve the full rigour of the professional scientific discipline.

We can identify three key levels of understanding in Computer Science:
\begin{itemize}
\item \textbf{Programming} is a threshold concept in Computer Science, as it introduces some of the basic challenges of the discipline.  
\item \textbf{Algorithms} involves studying solutions in computational terms: which solutions are better, in what circumstances, and why?
\item \textbf{Computability} is the study of \textit{problems} in 
computational terms: what can be computed, and why?
\end{itemize}

The goal of the PACT programme is to guide students through the key topics in programming and algorithms towards the ultimate goal of studying the process of computation itself.

The PACT group at NUIM has collaborated with teachers across 9 secondary schools to develop a flexible module which will engage Transition Year students.  The focus of the module is not on learning facts about computers but on developing creative ideas and new ways of thinking.  Continuing feedback will be used to expand the module into a full Junior Cycle short course. 

\section{Background}
The Lero group in the University of Limerick have established and education and outreach programme starting in 2007, and currently deliver Scratch-based material for primary and secondary schools\footnote{\url{http://www.scratch.ie/}}. Lero support teachers, students, parents and schools in Scratch as well as creating extensive lesson plans and teaching material for schools \cite{scratch2013}.


Bridge21\footnote{\url{http://www.bridge21.ie/}} is an education programme based in Trinity College which offers a new model of learning that can be adapted for use in secondary schools. They offer professional development workshops for teachers to train in certain areas \cite{conneely13}.

The Computing in Schools project in the United Kingdom looked at the current provision of education in Computing in UK schools, informed by evidence gathered from individuals and organisations with an interest in computing. One of their key findings was that the current delivery of Computing education in many UK schools is highly unsatisfactory. It also noted that there is a need to improve understanding in schools of the nature and scope of Computing. In particular there needs to be recognition that Computer Science is a rigorous academic discipline of great importance to the future careers of many pupils \cite{compSchools}.

\textbf{Need to put in more initiatives taking place around the world.}\\

\section{Experiment}
During the Summer of 2013 a number of schools were approached to gauge if there was interest in running such a pilot programme. The schools which came on board were:
\begin{compactitem}
  \item Maynooth Post Primary School, Co. Kildare.
  \item Pobalscoil Neasain, Baldoyle, Dublin 13.
  \item Confey Community College, Leixlip, Co. Kildare.
  \item Castelcomer Community School, Co. Kilkenny.
  \item Scoil Mhuire Community School, Clane, Co. Kildare.
  \item Jesus and Mary College, Goatstown, Dublin 14.
  \item Salesian College, Celbridge, Co. Kildare.
\end{compactitem}

At least one teacher from each of these schools agreed to participate in the pilot programme. The PACT team began to structure the content for the pilot programme and on the 25th of May and the 1st June 2013 two training sessions took place in NUIM. These sessions involved introducing the teachers to the content within the programme and also getting to know each other. It was hoped that a community of knowledge and learning would be created between all the teachers as they progressed with the pilot.

A Moodle system was hosted in the Department of Computer Science to store the content that was delivered in the training sessions. The content was divided in to five sections, namely:

\begin{compactenum}
  \item Introduction to Python I.
  \item Introduction to Python II.
  \item Algorithms.
  \item Graphics.
  \item Recursion and self-reference.
\end{compactenum}

Sections 1 and 2 provided an introduction to the Python programming language. These sections covered the basics of installation of Python and also looked at how to create and run programs written in Python. They looked at the main features of the language and gave sample programs for the teachers to use. An exercise book was created for all lessons within these sections which the teacher could use in class to get their students working in Python. In total 19 presentations were prepared for these sections which the teacher could break up into whatever format they felt worked for them. Section 3 focused on Algorithms and the process of writing them. It showed how to write a step by step procedure to solve particular problems. Section 4 looked at generating graphics using the Pygame set of modules. This section focused on allowing the participants to create fully featured games in the python language. Section 5 covers topics related to recursion which has been defined as ``the process of repeating items in a self-similar way" and is fundamental to the core theory of Computer Science. \\


\textbf{Should we put in learning outcomes here??}
\newline



Towards the end of the academic year 2013-14 we are going to host a competition which is open to all the students who were involved with the PACT pilot programme. This competition is aimed at providing examples of best practice and each school participating in the pilot is invited to select up to two teams, consisting of 3-5 students, to participate in the competition. The selection of teams and team members is entirely at the discretion of the schools. Participants in the competition should highlight explicitly how their experience with the PACT programme has contributed to two or more of the six key areas identified by the NCCA, namely:

\begin{compactenum}
  \item Managing Myself.
  \item Staying Well.
  \item Communicating.
  \item Being Creative.
  \item Working with Others.
  \item Managing Information and Thinking \cite{ncca2013}.
\end{compactenum}



\section{Future Work}
As of the 1st April 2014 we have already received interest from the following schools:
\begin{compactitem}
  \item Tallaght Community School, Dublin. 
  \item Sandford Park, Ranelagh, Dublin.
  \item St. Munchin's College, Limerick.
  \item Beneavin college, Finglas, Dublin.
\end{compactitem}

These schools have approached us directly from seeing our webpage or from teachers who are involved in the pilot programme this year. We hope to run this programme again for the academic year 2014-15 with a larger number of schools and take on board the feedback received from the teachers and students in the first year.

\section{Contact Us}
The PACT team can be contacted at the following email address: \url{pact@cs.nuim.ie}.
Our website is: \url{http://www.cs.nuim.ie/pact}.

\bibliographystyle{alpha}
\bibliography{ref}
\end{document}
