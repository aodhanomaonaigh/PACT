\documentclass{article}
\usepackage{graphicx}
\usepackage{color}
\usepackage{fancyhdr}

\begin{document}



\title{\large{\textbf{PACT: \\ Programming $^{\wedge}$ Algorithms $\Longrightarrow$ Computational Thinking}}}

\author{Aidan Mooney}

\begin{figure}
    \centering
    \includegraphics[width = 5cm]{logo}
    \label{logo}
\end{figure}
\maketitle

\section{Introduction}

On July 2nd the Maynooth University PACT team met to discuss SFI Discover Programme Funding and also a meeting that Rosemary had with the HEA and Department of Education (DoE).

It was agreed at this meeting that we would develop a Table of Contents for the PACT website (pact.cs.nuim.ie) that would be made available to the HEA and DoE to give them a sense of what was being done by the PACT team in this space. It was agreed that there would only be one live link to the Python Programming material, which is complete, and the other links and topics would link to an error page. However, the objective would be to show the different topics that would/are addressed by PACT.

In relation to the SFI Discover call it was proposed that we develop a "pop-up" Computational Thinking Stem Cell event. This could take place at events like \textit{Electric Picnic, Body and Soul, Taste of Dublin etc.} where there is a large audience of people available. This may be more beneficial to running an event in Maynooth (and/or other places) as there is a ready audience already there for us. It is envisaged that this event/festival would take groups of people, and possibly arrange them in age groups, where they would work on Computational Thinking problems using paper and pen. The Stem Cells would encompass all of the STEM subjects. In addition STEAM is a new buzz word which many say is the new STEM whereby Art is added to the traditional STEM subjects. As a side note Art was also mentioned in the SFI document so this is something that we should think about.

\end{document}
